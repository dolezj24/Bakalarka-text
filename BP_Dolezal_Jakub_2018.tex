% options:
% thesis=B bachelor's thesis
% thesis=M master's thesis
% czech thesis in Czech language
% slovak thesis in Slovak language
% english thesis in English language
% hidelinks remove colour boxes around hyperlinks

\documentclass[thesis=B,czech]{FITthesis}[2012/06/26]

\usepackage[utf8]{inputenc} % LaTeX source encoded as UTF-8

\usepackage{graphicx} %graphics files inclusion
% \usepackage{amsmath} %advanced maths
% \usepackage{amssymb} %additional math symbols

\usepackage{dirtree} %directory tree visualisation 
\RequirePackage{pdfpages}

% % list of acronyms
% \usepackage[acronym,nonumberlist,toc,numberedsection=autolabel]{glossaries}
% \iflanguage{czech}{\renewcommand*{\acronymname}{Seznam pou{\v z}it{\' y}ch zkratek}}{}
% \makeglossaries

\newcommand{\tg}{\mathop{\mathrm{tg}}} %cesky tangens
\newcommand{\cotg}{\mathop{\mathrm{cotg}}} %cesky cotangens

% % % % % % % % % % % % % % % % % % % % % % % % % % % % % % 
% ODTUD DAL VSE ZMENTE
% % % % % % % % % % % % % % % % % % % % % % % % % % % % % % 

\department{Katedra \ldots (DOPLŇTE)}
\title{Webový frontend informačního systému}
\authorGN{Jakub} %(křestní) jméno (jména) autora
\authorFN{Doležal} %příjmení autora
\authorWithDegrees{Jakub Doležal} %jméno autora včetně současných akademických titulů
\author{Jakub Doležal} %jméno autora bez akademických titulů
\supervisor{Ing. Jiří Chludil}
\acknowledgements{Doplňte, máte-li komu a za co děkovat. V~opačném případě úplně odstraňte tento příkaz.}
\abstractCS{Cílem této práce je návrh, implementace a otestování webového frontendu informačního systému pro firmu MAHALUX s.r.o.
	
	Na základě požadavků zadavatele je webový frontend implementovaný pomocí frameworku ReactJS v jazyce JavaScript. Serverová část aplikace je implementována v jazyce PHP s využitím frameworku Nette.
	
	Tento informační systém napomáhá k rychlému, hromadnému a bezchybnému zadávání dat. Zefektivňuje a centralizuje evidenci dat prováděnou zaměstnanci MAHALUX s.r.o.
	
	V uživatelské příručce jsou popsány složitější případy užití tohoto frontendu.
}
\abstractEN{Sem doplňte ekvivalent abstraktu Vaší práce v~angličtině.}
\placeForDeclarationOfAuthenticity{V~Praze}
\declarationOfAuthenticityOption{4} %volba Prohlášení (číslo 1-6)
\keywordsCS{webový frontend, informační systém pro MAHALUX s.r.o., návrh, implementace, testování, hromadné zadávání dat, ReactJS, PHP, framework Nette}
\keywordsEN{web frontend, information system, ReactJS, PHP, framework Nette}
% \website{http://site.example/thesis} %volitelná URL práce, objeví se v tiráži - úplně odstraňte, nemáte-li URL práce



\begin{document}

% \newacronym{CVUT}{{\v C}VUT}{{\v C}esk{\' e} vysok{\' e} u{\v c}en{\' i} technick{\' e} v Praze}
% \newacronym{FIT}{FIT}{Fakulta informa{\v c}n{\' i}ch technologi{\' i}}

\begin{introduction}
	V dnešní době je pro každou firmu rychlý a bezpečný IS nutností. Efektivní systém může přinést rozhodující náskok před konkurencí.
	Pro malou začínající firmu MAHALUX s.r.o. a její zaměstnance takový systém přinese potřebné centralizované ukládání dat a přístup k ním,
	což je mojí motivací pro výběr tohoto tématu.
	
	V práci se zabývám analýzou, návrhem, implementací a testováním webového informačního systému.
	Úkolem úvodní části práce je sběr funkčních a nefunkčních požadavků zadavatele. Dále také analýza složitějších případů užití.
	Následuje návrh struktury systému ve frameworku Nette, návrh komponetového prototypu a uživatelského rozhraní. Navazovat bude samotná implementace a na závěr testování.
	
	Systém bude využívat již zavedenou interní firemní databázi, která eviduje provozní data.
\end{introduction}

\chapter{Cíl práce}
	Cílem rešeršní části práce je analýza požadavků zákazníka a složitějších případů užití, prozkoumání existujících řešení webových informačních systémů a jejich frontendů.
	Cílem praktické části práce je návrh, implementace a otestování webového frontendu informačního systému.
\chapter{Analýza a návrh}

\section{Výběr modelovacího nástroje s ohledem na možnosti generování zdrojového kódu modelu}
	Nástroj s možností generování modelu může velice pomoci při vývoji. V první řadě je potřeba pro vizualizaci problémové domény. Za druhé lze některé takové nástroje použít pro generování zdrojového kódu modelu nebo pro vytvoření databázových DDL skriptů. Pro potřeby systému jsem zaobíral především možností generovat gettery a settery a možností definice ORM notace pro Doctrine 2.
\subsection{Enterprise Architect}
	Enterprise Architect je poměrně robustní UML nástroj pro modelování celého vývojového cyklu softwaru. Pokrývá vše od sběru požadavků, modelování business procesů, architektury, domény až po class diagramy, či databázové modely. Dále umožňuje generování zdrojových kódů modelu pro jazyky Java, C++, C\#, Python, PHP, aj., které provádí na základě modifikovatelných šablon. Tyto šablony využívají vlastní jazyk nazýváný Transformation Template language. Enterprise Architect také nabízí generování DDL skriptů pro mnoho databázových enginů jako jsou Oracle, MySql, SQL Server a další.

\section{research}
	Analýza požadavků:
	Zadavatel požaduje webový frontend, který napomůže k~rychlému a hromadnému zadávání provozních dat. Hlavním požadavkem je responzivní web, který pomůže odvést práci rychle a bezchybně.
	
	Realizované weby:	
	Jako dobrý vzor pro realizaci je webový portál iDoklad.cz. Především sekce pro správu faktur. Zahrnuje přehled vydaných faktur, jejich editaci a mnoho dalších funkcí.
	%Existující web is
	
	Technologie:
	Pro webové uživatelské rozhraní je použit framework ReactJS. Tento framework v jazyku JavaScript velmi snadno reaguje na změny dat ve formuláři a umožňuje rychle interaktivní překlesení požadovaných částí stránky.
	
	Serverová část aplikace je realizována v frameworku Nette využivající jazyk PHP. Tato část aplikace poskytuje API pro webové rozhraní a přistupuje k interní MySQL databázi.
	
	

\chapter{Realizace}

\section{praktická část}
	Webový frontend %dělá co jsme chtěli

\begin{conclusion}
	%sem napište závěr Vaší práce
	%shrnutí míry uspěšnosti realizace cíle a jak jsme toho dosáhli
	%výhled do budoucna - možné rozšíření výsledku práce
	Cílem práce byl návrh, implementace a otestování webového frontendu informačního systému.
	
	Aplikace byla nasazena. Splňuje požadavky zadané zadavatelem. Napomáhá rychle a bezchybně evidovat data. Je napojena na interní databázi kde ukládá zadávaná data.
\end{conclusion}

\bibliographystyle{csn690}
\bibliography{mybibliographyfile}

\appendix

\chapter{Seznam použitých zkratek}
% \printglossaries
\begin{description}
	\item[GUI] Graphical user interface
	\item[DDL] Data definition language
	\item[UML] Unified Modeling language
\end{description}


% % % % % % % % % % % % % % % % % % % % % % % % % % % % 
% % Tuto kapitolu z výsledné práce ODSTRAŇTE.
% % % % % % % % % % % % % % % % % % % % % % % % % % % % 
% 
% \chapter{Návod k~použití této šablony}
% 
% Tento dokument slouží jako základ pro napsání závěrečné práce na Fakultě informačních technologií ČVUT v~Praze.
% 
% \section{Výběr základu}
% 
% Vyberte si šablonu podle druhu práce (bakalářská, diplomová), jazyka (čeština, angličtina) a kódování (ASCII, \mbox{UTF-8}, \mbox{ISO-8859-2} neboli latin2 a nebo \mbox{Windows-1250}). 
% 
% V~české variantě naleznete šablony v~souborech pojmenovaných ve formátu práce\_kódování.tex. Typ může být:
% \begin{description}
% 	\item[BP] bakalářská práce,
% 	\item[DP] diplomová (magisterská) práce.
% \end{description}
% Kódování, ve kterém chcete psát, může být:
% \begin{description}
% 	\item[UTF-8] kódování Unicode,
% 	\item[ISO-8859-2] latin2,
% 	\item[Windows-1250] znaková sada 1250 Windows.
% \end{description}
% V~případě nejistoty ohledně kódování doporučujeme následující postup:
% \begin{enumerate}
% 	\item Otevřete šablony pro kódování UTF-8 v~editoru prostého textu, který chcete pro psaní práce použít -- pokud můžete texty s~diakritikou normálně přečíst, použijte tuto šablonu.
% 	\item V~opačném případě postupujte dále podle toho, jaký operační systém používáte:
% 	\begin{itemize}
% 		\item v~případě Windows použijte šablonu pro kódování \mbox{Windows-1250},
% 		\item jinak zkuste použít šablonu pro kódování \mbox{ISO-8859-2}.
% 	\end{itemize}
% \end{enumerate}
% 
% 
% V~anglické variantě jsou šablony pojmenované podle typu práce, možnosti jsou:
% \begin{description}
% 	\item[bachelors] bakalářská práce,
% 	\item[masters] diplomová (magisterská) práce.
% \end{description}
% 
% \section{Použití šablony}
% 
% Šablona je určena pro zpracování systémem \LaTeXe{}. Text je možné psát v~textovém editoru jako prostý text, lze však také využít specializovaný editor pro \LaTeX{}, např. Kile.
% 
% Pro získání tisknutelného výstupu z~takto vytvořeného souboru použijte příkaz \verb|pdflatex|, kterému předáte cestu k~souboru jako parametr. Vhodný editor pro \LaTeX{} toto udělá za Vás. \verb|pdfcslatex| ani \verb|cslatex| \emph{nebudou} s~těmito šablonami fungovat.
% 
% Více informací o~použití systému \LaTeX{} najdete např. v~\cite{wikilatex}.
% 
% \subsection{Typografie}
% 
% Při psaní dodržujte typografické konvence zvoleného jazyka. České \uv{uvozovky} zapisujte použitím příkazu \verb|\uv|, kterému v~parametru předáte text, jenž má být v~uvozovkách. Anglické otevírací uvozovky se v~\LaTeX{}u zadávají jako dva zpětné apostrofy, uzavírací uvozovky jako dva apostrofy. Často chybně uváděný symbol "{} (palce) nemá s~uvozovkami nic společného.
% 
% Dále je třeba zabránit zalomení řádky mezi některými slovy, v~češtině např. za jednopísmennými předložkami a spojkami (vyjma \uv{a}). To docílíte vložením pružné nezalomitelné mezery -- znakem \texttt{\textasciitilde}. V~tomto případě to není třeba dělat ručně, lze použít program \verb|vlna|.
% 
% Více o~typografii viz \cite{kobltypo}.
% 
% \subsection{Obrázky}
% 
% Pro umožnění vkládání obrázků je vhodné použít balíček \verb|graphicx|, samotné vložení se provede příkazem \verb|\includegraphics|. Takto je možné vkládat obrázky ve formátu PDF, PNG a JPEG jestliže používáte pdf\LaTeX{} nebo ve formátu EPS jestliže používáte \LaTeX{}. Doporučujeme preferovat vektorové obrázky před rastrovými (vyjma fotografií).
% 
% \subsubsection{Získání vhodného formátu}
% 
% Pro získání vektorových formátů PDF nebo EPS z~jiných lze použít některý z~vektorových grafických editorů. Pro převod rastrového obrázku na vektorový lze použít rasterizaci, kterou mnohé editory zvládají (např. Inkscape). Pro konverze lze použít též nástroje pro dávkové zpracování běžně dodávané s~\LaTeX{}em, např. \verb|epstopdf|.
% 
% \subsubsection{Plovoucí prostředí}
% 
% Příkazem \verb|\includegraphics| lze obrázky vkládat přímo, doporučujeme však použít plovoucí prostředí, konkrétně \verb|figure|. Například obrázek \ref{fig:float} byl vložen tímto způsobem. Vůbec přitom nevadí, když je obrázek umístěn jinde, než bylo původně zamýšleno -- je tomu tak hlavně kvůli dodržení typografických konvencí. Namísto vynucování konkrétní pozice obrázku doporučujeme používat odkazování z~textu (dvojice příkazů \verb|\label| a \verb|\ref|).
% 
% \begin{figure}\centering
% 	\includegraphics[width=0.5\textwidth, angle=30]{cvut-logo-bw}
% 	\caption[Příklad obrázku]{Ukázkový obrázek v~plovoucím prostředí}\label{fig:float}
% \end{figure}
% 
% \subsubsection{Verze obrázků}
% 
% % Gnuplot BW i barevně
% Může se hodit mít více verzí stejného obrázku, např. pro barevný či černobílý tisk a nebo pro prezentaci. S~pomocí některých nástrojů na generování grafiky je to snadné.
% 
% Máte-li například graf vytvořený v programu Gnuplot, můžete jeho černobílou variantu (viz obr. \ref{fig:gnuplot-bw}) vytvořit parametrem \verb|monochrome dashed| příkazu \verb|set term|. Barevnou variantu (viz obr. \ref{fig:gnuplot-col}) vhodnou na prezentace lze vytvořit parametrem \verb|colour solid|.
% 
% \begin{figure}\centering
% 	\includegraphics{gnuplot-bw}
% 	\caption{Černobílá varianta obrázku generovaného programem Gnuplot}\label{fig:gnuplot-bw}
% \end{figure}
% 
% \begin{figure}\centering
% 	\includegraphics{gnuplot-col}
% 	\caption{Barevná varianta obrázku generovaného programem Gnuplot}\label{fig:gnuplot-col}
% \end{figure}
% 
% 
% \subsection{Tabulky}
% 
% Tabulky lze zadávat různě, např. v~prostředí \verb|tabular|, avšak pro jejich vkládání platí to samé, co pro obrázky -- použijte plovoucí prostředí, v~tomto případě \verb|table|. Například tabulka \ref{tab:matematika} byla vložena tímto způsobem.
% 
% \begin{table}\centering
% 	\caption[Příklad tabulky]{Zadávání matematiky}\label{tab:matematika}
% 	\begin{tabular}{|l|l|c|c|}\hline
% 		Typ		& Prostředí		& \LaTeX{}ovská zkratka	& \TeX{}ovská zkratka	\tabularnewline \hline \hline
% 		Text		& \verb|math|		& \verb|\(...\)|	& \verb|$...$|		\tabularnewline \hline
% 		Displayed	& \verb|displaymath|	& \verb|\[...\]|	& \verb|$$...$$|	\tabularnewline \hline
% 	\end{tabular}
% \end{table}
% 
% % % % % % % % % % % % % % % % % % % % % % % % % % % % 

\chapter{Obsah přiloženého CD}

%upravte podle skutecnosti

\begin{figure}
	\dirtree{%
		.1 readme.txt\DTcomment{stručný popis obsahu CD}.
		.1 exe\DTcomment{adresář se spustitelnou formou implementace}.
		.1 src.
		.2 impl\DTcomment{zdrojové kódy implementace}.
		.2 thesis\DTcomment{zdrojová forma práce ve formátu \LaTeX{}}.
		.1 text\DTcomment{text práce}.
		.2 thesis.pdf\DTcomment{text práce ve formátu PDF}.
		.2 thesis.ps\DTcomment{text práce ve formátu PS}.
	}
\end{figure}

\end{document}
